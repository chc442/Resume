%-------------------------
% Resume in Latex
% Author : Hao Cheam
% License : MIT
% Last Updated: Aug 20, 2020 
%------------------------

% no periods at the end
%  move experience to before projects (once have relevant experience)

\documentclass[letterpaper,11pt]{article}

\usepackage{comment}
\usepackage{latexsym}
\usepackage[empty]{fullpage}
\usepackage{titlesec}
\usepackage{marvosym}
\usepackage[usenames,dvipsnames]{color}
\usepackage{verbatim}
\usepackage{enumitem}
\usepackage[hidelinks]{hyperref}
\usepackage{fancyhdr}
\usepackage[english]{babel}

\pagestyle{fancy}
\fancyhf{} % clear all header and footer fields
\fancyfoot{}
\renewcommand{\headrulewidth}{0pt}
\renewcommand{\footrulewidth}{0pt}

% Adjust margins
\addtolength{\oddsidemargin}{-0.5in}
\addtolength{\evensidemargin}{-0.5in}
\addtolength{\textwidth}{1in}
\addtolength{\topmargin}{-0.5in}
\addtolength{\textheight}{1.0in}

\urlstyle{same}

\raggedbottom
\raggedright
\setlength{\tabcolsep}{0in}

% Sections formatting
\titleformat{\section}{
  \vspace{-4pt}\scshape\raggedright\large
}{}{0em}{}[ \vspace{-5pt}]

 %Ruled
%\titleformat{\section}{
%  \vspace{-4pt}\scshape\raggedright\large
%}{}{0em}{}[\color{black}\titlerule \vspace{-5pt}]

%-------------------------
% Custom commands
\newcommand{\resumeItem}[2]{
  \item\small{
    \textbf{#1}{: #2 \vspace{-3pt}}
  }
}

% no bold text
\newcommand{\resumeItemAlt}[1]{
  \item\small {#1 \vspace{-3pt}
  }
}

\newcommand{\resumeSubheading}[4]{
  \vspace{-1pt} \item
    \begin{tabular*}{0.97\textwidth}[t]{l@{\extracolsep{\fill}}r}
      \textbf{#1} & #2  \\
      \textit{\small#3} & \textit{\small #4} \\
    \end{tabular*}\vspace{-5pt}
}

\newcommand{\JobSubheading}[4]{
  \vspace{-1pt}
    \begin{tabular*}{0.97\textwidth}[t]{l@{\extracolsep{\fill}}r}
      \textbf{#1} & #2  \\
      \textit{\small#3} & \textit{\small #4} \\
    \end{tabular*}\vspace{-5pt}
}

\newcommand{\EducationSubheading}[4]{
  \vspace{-1pt}
    \begin{tabular*}{\textwidth}[t]{l@{\extracolsep{\fill}}r}
      \textbf{#1} & #2  \\
      \textit{\small #3} & \textit{\small #4} \\
    \end{tabular*}\vspace{1pt}
}

\newcommand{\resumeSubItem}[2]{\resumeItem{#1}{#2}\vspace{-3.5pt}}
\newcommand{\resumeSubItemAlt}[1]{\resumeItemAlt{#1}\vspace{-3.5pt}}

\newcommand{\CertificationsSubItem}[1]{\resumeItemAlt{#1}\vspace{-3.5pt}}

\renewcommand{\labelitemii}{$\circ$}

\newcommand{\resumeSubHeadingListStart}{\begin{itemize}[leftmargin=*]}
\newcommand{\resumeSubHeadingListEnd}{\end{itemize}}
\newcommand{\resumeItemListStart}{\begin{itemize}}
\newcommand{\resumeItemListEnd}{\end{itemize}\vspace{-5pt}}

%-------------------------------------------
%%%%%%  CV STARTS HERE  %%%%%%%%%%%%%%%%%%%%%%%%%%%%


\begin{document}

%----------HEADING-----------------
\begin{center}
    \textbf{ \Large \color{BlueViolet} Hao Cheam}
\end{center}

\begin{tabular*}{\textwidth}{l@{\extracolsep{\fill}}r}
  \small \href{https://www.linkedin.com/in/haocheam/}{https://www.linkedin.com/in/haocheam/} & 
  \small \href{mailto:hcheam@umass.edu}{hcheam@umass.edu} \\
  \small \href{https://github.com/chc442}{https://github.com/chc442} & \small 413-313-8950
\end{tabular*}

% \mbox \newline

% --------SUMMARY--------------

%\section{\color{BlueViolet} Summary}
%\small Recent Master's graduate in Computer Science with 2 years of work experience involving very small teams and high degrees of autonomy. Heavy emphasis on Machine Learning for Master's coursework. Hands-on experience in traditional and machine learning-based computer vision methods, as well as embedded systems.
% \small \textit{Interested in robotics, computer vision, artificial intelligence, and embedded systems.}


%-----------EDUCATION-----------------
\section{\color{BlueViolet} Education}
    \EducationSubheading
      {University of Massachusetts Amherst}{}
      {MS in Computer Science;  GPA: 3.6/4.0}{Aug 2018 -- May 2020}
          {\small \textit{Courses: Computer Vision, Machine Learning, Robotics, Artificial Intelligence, Reinforcement Learning,  Neural Networks}}
          %  Secure Distributed Systems
    \vspace{5pt}
    \EducationSubheading
      {The University of Texas at Austin}{}
      {BS in Electrical and Computer Engineering (Honors; Minor in French);  GPA: 3.8/4.0}{Aug 2011 -- May 2015}
          % \resumeSubItemAlt{Relevant Courses: Video Processing, Image Processing, and Embedded Systems. }
          
     % Primary technical core: Signal processing and Networks; Secondary technical core: Embedded Systems and Computer Architecture.
     % Relevant coursework: Electronic circuits, Assembly programming, Embedded systems, Signals and Systems, Discrete Maths, Probability, Digital Logic, Real-time DSP Lab, Computer architecture, Image Processing, Machine Learning (Coursera), DSP Theory, Algorithms, Number Theory, Real Analysis.
     % Grad Courses: Estimation Theory, Probability, Video Processing.
  
%-----------EDUCATION END-----------------

  

%-----------PROJECTS START---------------------------------------
% Use STAR: situation, task, action, result
% Start bullet points with action verbs
% show results and tools you've used
% no more than 3 bullet points per project
\section{\color{BlueViolet} Projects}
    \small
    \textbf{Pokemon Classifier} 
    \vspace{-4pt}
        \begin{itemize}
            \item Trained a 7 layer Convolutional Neural Network (CNN) in Python using Keras and Tensorflow on a dataset of about 1400 pokemon images. Obtained accuracy of about 93\% over 6 classes
            \vspace{-5pt}
            \item The dataset was collected using the Microsoft Azure Bing Image Search API
        \end{itemize}
    \vspace{-2pt}
    
    \textbf{Mini Self Driving Car} 
    \vspace{-4pt}
        \begin{itemize}
            \item Wrote Dijkstra's shortest path algorithm in Python to solve path planning
            \vspace{-5pt}
            \item Implemented vision-based lane detection and vision-based PD controller in Python
            \vspace{-5pt}
            \item Project done on Arduino Uno and Raspberry Pi, in a team of 5 people
        \end{itemize}
    %For a mini self driving car group project, I wrote Dijkstra's shortest path algorithm in Python for our path planning module. I also wrote a simple computer vision solution in Python for lane detection. Programmed on an Arduino Uno and a Raspberry Pi
    \vspace{-2pt}
    
    \textbf{Non-Data-Driven Method of Learning Facets in Object Tracking} 
    \vspace{-4pt}
        \begin{itemize}
            \item Employed SIFT matching and mean-shift tracking on color histogram to track an object
            \vspace{-5pt}
            \item Facets are stored during tracking to augment the reference image database
            \vspace{-5pt}
            \item Allows tracking of an object after the object has been reintroduced into the scene while showing a different facet
        \end{itemize}
        % Given only one reference image, employed SIFT matching and mean-shift tracking on color
        % histogram to track the object. During tracking, its facets are stored to augment the reference image
        % database. This allows the user to track the object even when it is showing a different facet from the one in the reference image.
    \vspace{-2pt}
        
    \textbf{Blink Detection} 
    \vspace{-4pt}
        \begin{itemize}
            \item Used dlib's face detector and facial landmark models to localise the face and the eyes
            \vspace{-5pt}
            \item Computed the eye aspect ratio to determine whether a blink has occurred
        \end{itemize}
    \vspace{-2pt}
    
    \textbf{Grid Pathfinding with A* and Jump Point Search} 
    \vspace{-4pt}
        \begin{itemize}
            \item Implemented A* with Jump Point Search in C++ to solve a grid pathfinding problem
        \end{itemize}
    \vspace{-2pt}
    
    \textbf{High Confidence Policy Improvement (Reinforcement Learning)} 
    \vspace{-4pt}
        \begin{itemize}
            \item Using importance sampling, produced stochastic policies written in C++ that outperformed the behavior policy for an adaptive insulin advisor
        \end{itemize}
    \vspace{-2pt}
    
    \textbf{Voice Control of Magic Glass with Alexa} 
    \vspace{-4pt}
        \begin{itemize}
            \item Using Arduino code, interfaced an ESP8266 microcontroller (MCU) to communicate with Amazon Echo
            \vspace{-5pt}
            \item The MCU controls a relay that switches the magic glass
        \end{itemize}
        
    % unlisted projects
    \begin{comment}

    \small \textbf{Document Scanner} 
    \vspace{-4pt}
        \begin{itemize}
            \item Used perspective transform in Python and OpenCV to obtain a top-down view of a piece of document
        \end{itemize}
    \vspace{-2pt}
    
    \small \textbf{Color Based Object Tracking} 
    \vspace{-4pt}
        \begin{itemize}
            \item Using HSV thresholding, tracked a ball
            \vspace{-5pt}
            \item More than 30 fps
        \end{itemize}
    \vspace{-2pt}
    
    \resumeSubItem{Handwritten Digit Recognition}
      {Using HOG (Histogram of Oriented Gradients), SVM (Support Vector Machine), and OpenCV. On [placeholder] dataset, achieved [placeholder] accuracy}
      
    \resumeSubItem{Data Center Temperature Monitoring}
      {Python warning script for rising temperatures in data centres}
      
    \end{comment}

        
%-------------------PROJECTS END----------------------------


    

%--------------------------EXPERIENCE-----------------------------
\section{\color{BlueViolet} Experience}
    \vspace{-2pt}
    \JobSubheading
      {Powerware Systems}{Selangor, Malaysia}
      {Data Center Engineer}{Aug 2017 - Jun 2018}
      \vspace{0pt}
        \begin{itemize}
            \item Designed and built Data Center Monitoring Systems for temperature and humidity monitoring
            \vspace{-5pt}
            \item Created shop drawings of electrical panels in AutoCAD, planned cable routing, and built electrical panels
            \vspace{-5pt}
            \item Tools used: Modbus TCP/IP, BACNET, WAGO controllers, AutoCAD, MOXA controllers
            % automated tasks (programming, scrips), introduced better systems (version control, documentation)
        \end{itemize}
    \vspace{-2pt}

    \JobSubheading
      {ECE Department, The University of Texas at Austin}{Austin, TX}
      {Graduate Research Assistant}{Aug 2015 - Dec 2016}
      \vspace{0pt}
      \begin{itemize}
        \item Researched full-duplex radio. Ported MATLAB to C. Worked with USRP Ettus E110
        \vspace{-5pt}
        \item Focused on Interference Alignment via index coding and rank minimisation
      \end{itemize}
    \vspace{-2pt}
    
        
    % unlisted jobs
    \begin{comment}
    
    \JobSubheading
      {Pharos Lab, The University of Texas at Austin}{Austin, TX}
      {Lab Assistant}{Jan 2014 - May 2014}
      \vspace{0pt}
      \begin{itemize}
        \item Helped setup robots in Linux and ROS
      \end{itemize}
    
    \JobSubheading
      {Sanger Learning Center, The University of Texas at Austin}{Austin, TX}
      {Undergraduate Tutor}{Sep 2013 - Sep 2014}
      \vspace{0pt}
      \begin{itemize}
        \item Tutored peers in Calculus and French
      \end{itemize}
    
    \JobSubheading
      {ECE Department, The University of Texas at Austin}{Austin, TX}
      {Graduate Teaching Assistant}{Aug 2015 - Dec 2016}
      \vspace{0pt}
      \begin{itemize}
        \item Certified TA for undergraduate Signals and Systems course
      
    \end{comment}
 %------------------------EXPERIENCE END-------------------------------
      

%-----------------------------SKILLS----------------------------------------
\section{\color{BlueViolet} Skills}

      \small \textbf{Programming Languages:} Python, C, C++, MATLAB, Java
      % Octave
      \vspace{1pt}
      
      \textbf{Frameworks/Tools:} OpenCV, PyTorch, Keras, Linux, git, ROS
       % \LaTeX, AutoCAD, Code Composer Studio, LogicAid, SimuAid, Eclipse, Numpy, scikit-learn, Keil uVision, PCB Artist, Labview
      \vspace{1pt}

      % \textbf{Concepts/Algorithms:} Deep learning, Particle filter, SLAM, Reinforcement learning, Object tracking
      % A* pathfinding, CNN for images, machine learning algorithms for regression/classification, object detection, neural networks
      \vspace{1pt}
      
      % \textbf{Misc:} Embedded system prototyping, Soldering
      % tutoring
      % \vspace{2pt}
      
      %   \resumeSubItem{Technologies}
      %   {AWS, Play, React, Kafka, GCE}

      \textbf{Languages:} Chinese (Fluent), Malay (Intermediate), French (Intermediate)

%----------------------------SKILLS END-----------------------------------



%--------CERTIFICATES AND AWARDS---------------------------------
\section{\color{BlueViolet} Certifications}
  \resumeSubHeadingListStart
  
      \CertificationsSubItem {Business Foundations Series for Scientists and Engineers (8 weeks, Summer 2019)}
      
      % unlisted
      \begin{comment}
      
      \resumeSubItem{Awards}
      {1st Place Team in Malaysian Math Olympiad (2009), Silver Prize in Australian Mathematics Competition (2009), 2nd Place in high school English Essay Writing Competition (2010), Semi-finalist in Malaysian National Classical Guitar Competition (2010).}
     
      % Malaysian National Physics Competition, 2010, 2nd-place in individual category, 1st-place in group category
      % Australian Mathematics Competition, 2009, Silver Prize
      % Malaysian National Mathematics Olympiad, 2009, 1st-place in group category
      % Semi-finalist in 2010 Malaysian National Classical Guitar Competition
      % 2nd place in Chong Hwa Independent High School English Essay Writing Competition
      
      \end{comment}
      
      % ----scholarships (unlisted)----
      \begin{comment}
      
      \resumeSubItem{Scholarships}
      {Stuff}
      
      \end{comment}
      
  \resumeSubHeadingListEnd


%----------------CERTIFICATES AND AWARDS END---------------------------

\end{document}